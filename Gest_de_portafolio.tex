\documentclass[11pt]{article}
\usepackage{sectsty}
\usepackage{graphicx}
\usepackage{multirow}
\usepackage[left= 2cm]{geometry}
\title{Especificación de requisitos de software}
\author{Genesis Anjara Rivas Cardenas}
\date{September 2023}
\title{ Gestion de portafolio }
\author { Aguilar Jordie, Bagui Carlos, Rivas Genesis, Mero Alexis, Toninho Monroy }
\date{\today}
\begin{document}

\newpage

\vspace{2cm} 
\begin{center}
\Huge 
\textbf{ESPECIFICACIÓN DE REQUISITOS DE SOFTWARE
PROYECTO AUTOMATIZACIÓN GESTION DE
PORTAFOLIOS}
\end{center}

\newpage

\Huge
\textbf{Ficha del documento}
\vspace{7cm}

\normalsize

\begin{tabular}{|c|c|c|c|}
\hline
Fecha & Revisión & Líder del Proyecto & Verificación del Proyecto \\
\hline
\multirow{8}{*}{20/10/2023} & \multirow{8}{*}{1.1} & \multirow{8}{*}{Mero Arboleda Alexis Gabriel} & \multirow{8}{*}{ING.Stalin Francis} \\
& & & \\
& & & \\
& & & \\
& & & \\
& & & \\
& & & \\
& & & \\
\hline
\end{tabular}

\newpage


\section{\textbf{Introducción}}

Este documento es una Especificación de Requisitos de Software (ERS)
para el Sistema de Gestión de Portafolios. La estructura de esta 
especificación se basa en las directrices proporcionadas por el
estándar IEEE Práctica Recomendada para Especificaciones de 
Requisitos de Software ANSI/IEEE 830, 1998.

\subsection{\textbf{Propósito}}

El propósito de este documento es definir las especificaciones 
funcionales y no funcionales para el desarrollo de un sistema de
gestión de portafolios, el cual será utilizado por instituciones
educativas, profesores y estudiantes. Este sistema tiene como
objetivo principal facilitar la gestión eficiente de portafolios,
incluyendo la planificación, la comunicación y la evaluación.

\subsection{\textbf{Alcance}}

Esta especificación de requisitos está dirigida a usuarios y 
administradores del sistema de gestión de portafolios. El sistema se 
enfoca en la gestión de portafolios en el ámbito educativo, lo que 
incluye la planificación de programas académicos, la comunicación 
entre profesores y estudiantes, y la evaluación del desempeño 
académico. El software permitirá una gestión efectiva de las 
portafolios, promoviendo un entorno de aprendizaje enriquecedor y 
eficiente.
\subsection{\textbf{Personal involucrado}}
\vspace{10pt}

\begin{tabular}{|c|c|}

\hline
Nombre & Alexis Gabriel Mero Arboleda  \\
\hline
Rol & Analista, arquitecto y desarrollador de software \\
\hline
Responsabilidad & Análisis de información, arquitectura y programación del Software  \\
\hline
Información de contacto & alexis.mero@utelvt.edu.ec \\
\hline

\end{tabular}

\vspace{10pt}

\begin{tabular}{|c|c|}

\hline
Nombre & Genesis Anjara Rivas Cardenas  \\
\hline
Rol & Analista, arquitecto y desarrollador de software \\
\hline
Responsabilidad & Análisis de información, arquitectura y programación del Software  \\
\hline
Información de contacto & genesis.rivas.cardenas@utelvt.edu.ec \\
\hline

\end{tabular}
\vspace{10pt}

\begin{tabular}{|c|c|}

\hline
Nombre & Carlos Steven Bagui Ortiz  \\
\hline
Rol & Analista, arquitecto y desarrollador de software \\
\hline
Responsabilidad & Análisis de información, arquitectura y programación del Software  \\
\hline
Información de contacto & carlos.bagui.ortiz@utelvt.edu.ec \\
\hline

\end{tabular}
\vspace{10pt}

\begin{tabular}{|c|c|}

\hline
Nombre & Toninho Paublo Monroy Ulloa  \\
\hline
Rol & Analista, arquitecto y desarrollador de software \\
\hline
Responsabilidad & Análisis de información, arquitectura y programación del Software  \\
\hline
Información de contacto & Toninho.monroy.ulloa@utelvt.edu.ec \\
\hline

\end{tabular}
\vspace{10pt}

\begin{tabular}{|c|c|}

\hline
Nombre & Jordie Jorell Aguilar Martinez \\
\hline
Rol & Analista, arquitecto y desarrollador de software \\
\hline
Responsabilidad & Análisis de información, arquitectura y programación del Software  \\
\hline
Información de contacto & Jordie.aquilar.martinez@utelvt.edu.ec \\
\hline

\end{tabular}
\vspace{10pt}

\subsection{\textbf{Definiciones, acrónimos y abreviaturas}}

\vspace{10pt}

\begin{tabular}{|c|c|}

\hline
\textit{Nombre} & \textit{Descripción}  \\
\hline
Usuario & Persona que usará el sistema para gestionar portafolios \\
\hline
SGAI & Sistema de Gestión de Portafolios Integrado \\
\hline
ERS & Especificación de Requisitos Software \\
\hline
RF & Requerimiento Funcional \\
\hline
RNF & Requerimiento No Funcional\\
\hline
FEE & Factibilidad, exactitud y eficacia \\
\hline

\end{tabular}

\subsection{\textbf{Referencias}}

\begin{tabular}{|c|c|}

\hline
\textit{Título de documento} & \textit{Referencias}  \\
\hline
Standard IEEE 830 - 1998 & IEEE-830 \\
\hline

\end{tabular}
