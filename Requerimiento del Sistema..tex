\documentclass[11pt]{article}
\usepackage{sectsty}
\usepackage{graphicx}
\usepackage{multirow}
\usepackage[left= 2cm]{geometry}
\title{Especificación de requisitos de software}
\author{Genesis Anjara Rivas Cardenas}
\date{September 2023}
\title{ Gestion de portafolio }
\author { Aguilar Jordie, Bagui Carlos, Rivas Genesis, Mero Alexis, Toninho Monroy }
\date{\today}
\begin{document}

\newpage

\vspace{2cm} 
\begin{center}
\Huge 
\textbf{ESPECIFICACIÓN DE REQUISITOS DE SOFTWARE
PROYECTO AUTOMATIZACIÓN GESTION DE
PORTAFOLIOS}
\end{center}

\newpage

\Huge
\textbf{Ficha del documento}
\vspace{7cm}

\normalsize

\begin{tabular}{|c|c|c|c|}
\hline
Fecha & Revisión & Líder del Proyecto & Verificación del Proyecto \\
\hline
\multirow{8}{*}{20/10/2023} & \multirow{8}{*}{1.1} & \multirow{8}{*}{Mero Arboleda Alexis Gabriel} & \multirow{8}{*}{ING.Stalin Francis} \\
& & & \\
& & & \\
& & & \\
& & & \\
& & & \\
& & & \\
& & & \\
\hline
\end{tabular}

\newpage


\section{\textbf{Introducción}}

Este documento es una Especificación de Requisitos de Software (ERS)
para el Sistema de Gestión de Portafolios. La estructura de esta 
especificación se basa en las directrices proporcionadas por el
estándar IEEE Práctica Recomendada para Especificaciones de 
Requisitos de Software ANSI/IEEE 830, 1998.

\subsection{\textbf{Propósito}}

El propósito de este documento es definir las especificaciones 
funcionales y no funcionales para el desarrollo de un sistema de
gestión de portafolios, el cual será utilizado por instituciones
educativas, profesores y estudiantes. Este sistema tiene como
objetivo principal facilitar la gestión eficiente de portafolios,
incluyendo la planificación, la comunicación y la evaluación.

\subsection{\textbf{Alcance}}

Esta especificación de requisitos está dirigida a usuarios y 
administradores del sistema de gestión de portafolios. El sistema se 
enfoca en la gestión de portafolios en el ámbito educativo, lo que 
incluye la planificación de programas académicos, la comunicación 
entre profesores y estudiantes, y la evaluación del desempeño 
académico. El software permitirá una gestión efectiva de las 
portafolios, promoviendo un entorno de aprendizaje enriquecedor y 
eficiente.
\subsection{\textbf{Personal involucrado}}
\vspace{10pt}

\begin{tabular}{|c|c|}

\hline
Nombre & Alexis Gabriel Mero Arboleda  \\
\hline
Rol & Analista, arquitecto y desarrollador de software \\
\hline
Responsabilidad & Análisis de información, arquitectura y programación del Software  \\
\hline
Información de contacto & alexis.mero@utelvt.edu.ec \\
\hline

\end{tabular}

\vspace{10pt}

\begin{tabular}{|c|c|}

\hline
Nombre & Genesis Anjara Rivas Cardenas  \\
\hline
Rol & Analista, arquitecto y desarrollador de software \\
\hline
Responsabilidad & Análisis de información, arquitectura y programación del Software  \\
\hline
Información de contacto & genesis.rivas.cardenas@utelvt.edu.ec \\
\hline

\end{tabular}
\vspace{10pt}

\begin{tabular}{|c|c|}

\hline
Nombre & Carlos Steven Bagui Ortiz  \\
\hline
Rol & Analista, arquitecto y desarrollador de software \\
\hline
Responsabilidad & Análisis de información, arquitectura y programación del Software  \\
\hline
Información de contacto & carlos.bagui.ortiz@utelvt.edu.ec \\
\hline

\end{tabular}
\vspace{10pt}

\begin{tabular}{|c|c|}

\hline
Nombre & Toninho Paublo Monroy Ulloa  \\
\hline
Rol & Analista, arquitecto y desarrollador de software \\
\hline
Responsabilidad & Análisis de información, arquitectura y programación del Software  \\
\hline
Información de contacto & Toninho.monroy.ulloa@utelvt.edu.ec \\
\hline

\end{tabular}
\vspace{10pt}

\begin{tabular}{|c|c|}

\hline
Nombre & Jordie Jorell Aguilar Martinez \\
\hline
Rol & Analista, arquitecto y desarrollador de software \\
\hline
Responsabilidad & Análisis de información, arquitectura y programación del Software  \\
\hline
Información de contacto & Jordie.aquilar.martinez@utelvt.edu.ec \\
\hline

\end{tabular}
\vspace{10pt}

\subsection{\textbf{Definiciones, acrónimos y abreviaturas}}

\vspace{10pt}

\begin{tabular}{|c|c|}

\hline
\textit{Nombre} & \textit{Descripción}  \\
\hline
Usuario & Persona que usará el sistema para gestionar portafolios \\
\hline
SGAI & Sistema de Gestión de Portafolios Integrado \\
\hline
ERS & Especificación de Requisitos Software \\
\hline
RF & Requerimiento Funcional \\
\hline
RNF & Requerimiento No Funcional\\
\hline
FEE & Factibilidad, exactitud y eficacia \\
\hline

\end{tabular}

\subsection{\textbf{Referencias}}

\begin{tabular}{|c|c|}

\hline
\textit{Título de documento} & \textit{Referencias}  \\
\hline
Standard IEEE 830 - 1998 & IEEE-830 \\
\hline
\end{tabular}

\begin{document}

\newpage
\subsection{\textbf{Resumen}}
Este documento sobre el software de gestión de
portafolio consta de tres secciones. En la primera
sección se realiza una introducción al mismo y se
proporciona un visón general de la especificación
de requisitos del software de gestión de portafolio.
En la segunda sesión del documento se realiza una
descripción general del sistema, con el fin de
conocer las principales funciones que este debe
realizar.  
Por último, la tercera sección del documento es
aquella en la que se definen detalladamente los
requisitos que debe satisfacer el sistema. Esta
sección se centra en los aspectos no funcionales
del sistema y cómo debe comportarse en términos de
rendimiento, seguridad y calidad

\section{\textbf{Descripción general}}
\subsection{\textbf{perspectiva del producto}}

El sistema de la gestión de portafolio está
enfocado a cubrir los requerimientos de la empresa,
una plataforma de gestión de portafolio sólida y
efectiva que satisfaga las necesidades de los
usuarios y proporcione un valor significativo en la
gestión de sus inversiones. La colaboración cercana
con los usuarios y la retroalimentación constante
son esenciales para garantizar que el producto
cumpla con sus expectativas y requisitos.

\subsection{\textbf{Funcionalidad del Software}}
\includegraphics[scale=0.6]{modelo_grafico (1).jpg}

\subsection{\textbf{Características de los
usuarios}}

\vspace{10pt}

\begin{tabular}{|c|c|}

\hline
Tipo de usuario  & Usuario del Software \\
\hline
Formación & Universitaria \\
\hline
Actividades & Control y manejo del sistema en
general  \\
\hline

\end{tabular}

\subsection{\textbf{Restricciones}}

\begin{enumerate}
\item 	Lenguaje y tecnologías en uso: java, PHPM y
Admin.
\item  tiempo
\item disponibilidad de datos en tiempo real
\item 	Cumplimiento normativo
\item  El sistema deberá tener un diseño e
implementación sencilla.
\end{enumerate}

\subsection{\textbf{Suposiciones y dependencias}}
\begin{enumerate}
\item Se asume que los requisitos aquí descritos
son estables.
\item Los equipos en los que se vaya a ejecutar el
sistema deben cumplir los requisitos antes
indicados para garantizar una ejecución
correcta de la misma.
\end{enumerate}
\section{\textbf{Descripción general}}

\subsection{\textbf{Requerimientos Funcionales}}
\vspace{15pt}
\begin{tabular}{|c|c|} 
\hline
\textbf{Identificación del requerimiento:} & RF01 
\\
\hline
\textbf{Nombre del Requerimiento}: &       
Interfaz de usuario \\
\hline
\textbf{Características:} & Los usuario deberán
identificarse para acceder al sistema  \\
\hline
\textbf{Descripción del requerimiento:} & El
sistema podrá brindar 
datos de visualización\\ & al usuario autorizado
para que constate 
la actividad de los portafolios \\
\hline
\textbf{Requerimiento NO funcional:} & RNF01 \\
& RNF04  \\
\hline
\textbf{Prioridad de requerimiento:} & Alta \\
\hline
\end{tabular}

\vspace{15pt}
\begin{tabular}{|c|c|} 
\hline
\textbf{Identificación del requerimiento:} & RF02 
\\
\hline
\textbf{Nombre del Requerimiento}: &Seguridad de
datos \\
\hline
\textbf{Características:} & Es fundamental para
proteger \\ & la información financiera y personal
de los usuarios \\
\hline
\textbf{Descripción del requerimiento:} & El
sistema podrá brindar 
datos de visualización al \\ & usuario autorizado
para que 
monitorice los documentos cargados\\
\hline
\textbf{Requerimiento NO funcional:} & RNF01 \\
\hline
\textbf{Prioridad de requerimiento:} & Alta \\
\hline

\end{tabular}

\vspace{15pt}
\begin{tabular}{|c|c|} 
\hline
\textbf{Identificación del requerimiento:} & RF03 
\\
\hline
\textbf{Nombre del Requerimiento}: & Notificaciones
y alertas  \\
\hline
\textbf{Características:} & La capacidad de
configurar y \\ & recibir notificaciones y alertas
personalizadas \\ & es una característica
importante en la gestión de portafolio \\
\hline
\textbf{Descripción del requerimiento:} & Los
usuarios deben tener \\ & la capacidad de
personalizar \\ & las notificaciones y alertas
según sus preferencias \\
\hline
\textbf{Requerimiento NO funcional:} & RNF01 \\
& RNF04  \\
\hline
\textbf{Prioridad de requerimiento:} & Alta \\
\hline
\end{tabular}

\vspace{15pt}

\begin{tabular}{|c|c|} 
\hline
\textbf{Identificación del requerimiento:} & RF04 
\\
\hline
\textbf{Nombre del Requerimiento}: & Gestión de
riesgos\\
\hline
\textbf{Características:} & identificación de los
riesgos \\ & potenciales que  pueden afectar  el
desempeño del portafolio.\\
\hline
\textbf{Descripción del requerimiento:} & Los
riesgos pueden incluir\\ & factores económicos,
políticos y de crédito, entre otros. \\
\hline
\textbf{Requerimiento NO funcional:} & RNF01  \\
\hline
\textbf{Prioridad de requerimiento:} & Alta \\
\hline

\end{tabular}

\vspace{15pt}

\begin{tabular}{|c|c|} 
\hline
\textbf{Identificación del requerimiento:} & RF05 
\\
\hline
\textbf{Nombre del Requerimiento}: &Asignación de
recursos  \\
\hline
\textbf{Características:} & La asignación de
recursos busca diversificar \\ & la inversión en
diferentes clases de activos \\ & o estrategias
de\\ & inversión para reducir el riesgo y maximizar
el rendimiento \\
\hline
\textbf{Descripción del requerimiento:} & Aquí se
asignan los recursos \\ & identificados a las
tareas o proyectos correspondientes \\
\hline
\textbf{Requerimiento NO funcional:} & RNF01 \\
& RNF02  \\
\hline
\textbf{Prioridad de requerimiento:} & Alta \\
\hline
\end{tabular}
\newpage




\vspace{15pt}

\begin{tabular}{|c|c|} 
\hline
\textbf{Identificación del requerimiento:} & RF06  \\
\hline
\textbf{Nombre del Requerimiento}: & Modificar \\
\hline
\textbf{Características:} & El Sistema de Gestión de portafolio
permitirá al administrador \\ & modificar  \\ & los datos 
personales del cliente, usuario e instructores. \\
\hline
\textbf{Descripción del requerimiento:} & Permite al administrador 
modificar datos de los usuarios\\ & clientes,instructores,datos 
personales y ventas, \\ & que lleve a cabo el Software. \\ 
\hline
\textbf{Requerimiento NO funcional:} & RNF01 \\
& RNF02  \\ & RNF05 \\
\hline
\textbf{Prioridad de requerimiento:} & Alta \\
\hline

\end{tabular}


\vspace{15pt}

\begin{tabular}{|c|c|} 
\hline
\textbf{Identificación del requerimiento:} & RF07  \\
\hline
\textbf{Nombre del Requerimiento}: & Gestionar portafolio. \\
\hline
\textbf{Características:} & El sistema permitirá la gestión del 
portafolio. \\ 
\hline
\textbf{Descripción del requerimiento:} & Permite al administrador 
generar informes  sobre el portafolio \\ &   requerido por el 
usuario.\\ 
\hline
\textbf{Requerimiento NO funcional:} & RNF01 \\
& RNF02  \\
\hline
\textbf{Prioridad de requerimiento:} & Alta \\
\hline

\end{tabular}


\vspace{15pt}

\begin{tabular}{|c|c|} 
\hline
\textbf{Identificación del requerimiento:} & RF08  \\
\hline
\textbf{Nombre del Requerimiento}: & registro del portafolio del
usuario \\
\hline
\textbf{Características:} & Permitirá que se registren datos de los 
perfiles autorizados \\ & para esta gestión de portafolio \\
\hline
\textbf{Descripción del requerimiento:} & Permitira \\ & al usuario 
la 
capacidad de agregar informacion al portafolio \\ 
\hline
\textbf{Requerimiento NO funcional:} & RNF01 \\
\hline
\textbf{Prioridad de requerimiento:} & Alta \\
\hline


\end{tabular}

\subsection{\textbf{Requerimientos No Funcionales}}

\vspace{15pt}

\begin{tabular}{|c|c|} 
\hline
\textbf{Identificación del requerimiento:} & RNF01  \\
\hline
\textbf{Nombre del Requerimiento}: & Agregar información imprecisa 
\\ & Sistema de gestión de portafolio. \\
\hline
\textbf{Características:} & El Sistema presentara información que 
no sea requerida \\ & por el de usuario \\ & al momento de 
interactuar con el sistema de gestión de portafolio.
\\
\hline
\textbf{Descripción del requerimiento:} & El sistema debe tener una
interfaz de uso intuitiva y sencilla.  \\
\hline
\textbf{Prioridad de requerimiento:} & baja \\
\hline

\end{tabular}


\vspace{15pt}

\begin{tabular}{|c|c|} 
\hline
\textbf{Identificación del requerimiento:} & RNF02  \\
\hline
\textbf{Nombre del Requerimiento}: & Ayuda al usuario en el uso del 
\\ & 
Sistema de gestión de Portafolio \\
\hline
\textbf{Características:} & Permite que el usuario consulte 
informacion del portafolio\\
\hline
\textbf{Descripción del requerimiento:} & La interfaz del sistema 
no contará con ningún \\ & sistema de ayuda o soporte para los 
usuarios,\\ & lo que dificultará  la comprensión y el uso eficiente 
del sistema \\ & de gestión de portafolio.\\
\hline
\textbf{Prioridad de requerimiento:} & baja \\
\hline

\end{tabular}

\vspace{15pt}

\begin{tabular}{|c|c|} 
\hline
\textbf{Identificación del requerimiento:} & RNF03  \\
\hline
\textbf{Nombre del Requerimiento}: & Mantenimiento del sistema de 
\\ & gestión de portafolio. \\
\hline
\textbf{Características:} & El sistema de gestión de portafolio no 
tendrá un manual de uso \\ & y manual de usuario para facilitar los 
mantenimientos \\ & que serán realizados por el administrador.\\
\hline
\textbf{Descripción del requerimiento:} & El sistema de gestión de 
portafolio no \\ & puede disponer de una documentación fácilmente 
actualizable que \\ & permita gestionar y dar mantenimiento \\ & 
con el menor esfuerzo posible.\\
\hline
\textbf{Prioridad de requerimiento:} & baja \\
\hline

\end{tabular}

\vspace{15pt}

\begin{tabular}{|c|c|} 
\hline
\textbf{Identificación del requerimiento:} & RNF04  \\
\hline
\textbf{Nombre del Requerimiento}: & Diseño de la interfaz a la 
característica del sistema. \\
\hline
\textbf{Características:} & El Sistema  no tendrá una interfaz de 
usuario,\\ & teniendo en cuenta las características \\ & de la 
gestión de portafolio.\\
\hline

\textbf{Descripción del requerimiento:} & La interfaz de usuario no 
se ajustara a las características \\ & del portafolio ,dentro de lo 
cual se incorporara el sistema \\ & de gestión de portafolio y el 
inventario. \\
\hline
\textbf{Prioridad de requerimiento:} & baja \\
\hline

\end{tabular}
\end{tabular}

\vspace{15pt}

\begin{tabular}{|c|c|} 
\hline
\textbf{Identificación del requerimiento:} & RNF05  \\
\hline
\textbf{Nombre del Requerimiento}: & Diseño de la interfaz a la 
característica del sistema. \\
\hline
\textbf{Características:} &  El sistema de gestión de portafolio no garantizara un desempeño  \\ & en cuanto a los datos almacenados en el sistema ofreciéndole \\ &  una confiabilidad a esta misma.\\
\hline

\textbf{Descripción del requerimiento:} & No se garantizara el desempeño del sistema de gestión \\ & de portafolio a los diferentes usuarios . En este sentido \\ & la información almacenada o registros realizados \\ &  podrán ser consultados y actualizados permanentemente \\ &  y simultáneamente, sin que afecte el tiempo de respuesta. \\
\hline
\textbf{Prioridad de requerimiento:} & baja \\
\hline

\end{tabular}
\end{tabular}

\vspace{15pt}

\begin{tabular}{|c|c|} 
\hline
\textbf{Identificación del requerimiento:} & RNF06  \\
\hline
\textbf{Nombre del Requerimiento}: & Acceso al portafolio \\
\hline
\textbf{Características:} & No garantiza al usuario el acceso al portafolio de acuerdo  \\ & a la información que posea.\\
\hline

\textbf{Descripción del requerimiento:} & No tendrá facilidad e 
instrucciones  para  permitir el acceso a la  \\ & gestión de 
portafolio   al personal autorizado\\ & a través del sistema con la 
intención   de consultar \\ & y subir información pertinente de 
cada uno. \\
\hline
\textbf{Prioridad de requerimiento:} & baja \\
\hline

\end{tabular}
