\documentclass{article}
\usepackage{graphicx} % Required for inserting images
\usepackage[left=2cm]{geometry}
\usepackage[left=2cm]{geometry}
\usepackage{multirow}
\title{Especificación de requisitos de software}
\author{Genesis Anjara Rivas Cardenas}
\date{September 2023}

\begin{document}
\documentclass[11pt]{article} 
\usepackage{sectsty}
\usepackage{graphicx}

\title{ Gestion de portafolio }
\author { Aguilar Jordie, Bagui Carlos, Rivas Genesis, Mero Alexis, Toninho Monroy }
\date{\today}
\begin{document}
\maketitle  
\section{Introduccion}

Introducción:

La presente investigación se refiere al tema de la Gestión de Portafolios.
La característica principal de este tipo de portafolios se centra en la calidad por encima de la cantidad, por ello, tienes que dar prioridad a aquellos trabajos que demuestran lo mejor de tus cualidades profesionales.
Para analizar esta auditoria es necesario mencionar una de sus causas. Una de ellas es la falta de, desenvolvimiento en la organización de datos que conlleve a la armazón   de información.

\section{Objetivo General}


Objetivo general:

Ajustar los documentos solicitados, obserbar el progreso y adáptarlo continuamente a los cambios para garantizar la entrega oportuna de un software de alta calidad.

\section{Objetivo Especifico}

Objetivo especifico
Asegurar de que los documentos seleccionados se adapten a la organización del portafolio.
Control y Monitoreo que también se encarga de controlar y monitoriar el progreso de documentos en el portafolio.
\newpage

\vspace{2cm} 
\begin{center}
\Huge 
\textbf{ESPECIFICACIÓN DE REQUISITOS DE SOFTWARE
PROYECTO AUTOMATIZACIÓN GESTION DE
PORTAFOLIOS}
\end{center}

\newpage

\Huge
\textbf{Ficha del documento}
\vspace{7cm}

\normalsize

\begin{tabular}{|c|c|c|c|}
\hline
Fecha & Revisión & Líder del Proyecto & Verificación del Proyecto \\
\hline
\multirow{8}{}{20/10/2023} & \multirow{8}{}{1.1} & \multirow{8}{}{Mero Arboleda Alexis Gabriel} & \multirow{8}{}{ING.Stalin Francis} \\
& & & \\
& & & \\
& & & \\
& & & \\
& & & \\
& & & \\
& & & \\
\hline
\end{tabular}

\newpage


\section{\textbf{Introducción}}

Este documento es una Especificación de Requisitos de Software (ERS)
para el Sistema de Gestión de Portafolios. La estructura de esta 
especificación se basa en las directrices proporcionadas por el
estándar IEEE Práctica Recomendada para Especificaciones de 
Requisitos de Software ANSI/IEEE 830, 1998.

\subsection{\textbf{Propósito}}

El propósito de este documento es definir las especificaciones 
funcionales y no funcionales para el desarrollo de un sistema de
gestión de asignaturas, el cual será utilizado por instituciones
educativas, profesores y estudiantes. Este sistema tiene como
objetivo principal facilitar la gestión eficiente de asignaturas,
incluyendo la planificación, la comunicación y la evaluación.

\subsection{\textbf{Alcance}}

Esta especificación de requisitos está dirigida a usuarios y 
administradores del sistema de gestión de asignaturas. El sistema se 
enfoca en la gestión de asignaturas en el ámbito educativo, lo que 
incluye la planificación de programas académicos, la comunicación 
entre profesores y estudiantes, y la evaluación del desempeño 
académico. El software permitirá una gestión efectiva de las 
asignaturas, promoviendo un entorno de aprendizaje enriquecedor y 
eficiente.
\subsection{\textbf{Personal involucrado}}
\vspace{10pt}

\begin{tabular}{|c|c|}

\hline
Nombre & Alexis Gabriel Mero Arboleda  \\
\hline
Rol & Analista, arquitecto y desarrollador de software \\
\hline
Responsabilidad & Análisis de información, arquitectura y programación del Software  \\
\hline
Información de contacto & alexis.mero@utelvt.edu.ec \\
\hline

\end{tabular}

\vspace{10pt}

\begin{tabular}{|c|c|}

\hline
Nombre & Genesis Anjara Rivas Cardenas  \\
\hline
Rol & Analista, arquitecto y desarrollador de software \\
\hline
Responsabilidad & Análisis de información, arquitectura y programación del Software  \\
\hline
Información de contacto & genesis.rivas.cardenas@utelvt.edu.ec \\
\hline

\end{tabular}
\vspace{10pt}

\begin{tabular}{|c|c|}

\hline
Nombre & Carlos Steven Bagui Ortiz  \\
\hline
Rol & Analista, arquitecto y desarrollador de software \\
\hline
Responsabilidad & Análisis de información, arquitectura y programación del Software  \\
\hline
Información de contacto & carlos.bagui.ortiz@utelvt.edu.ec \\
\hline

\end{tabular}
\vspace{10pt}

\begin{tabular}{|c|c|}

\hline
Nombre & Toninho Paublo Monroy Ulloa  \\
\hline
Rol & Analista, arquitecto y desarrollador de software \\
\hline
Responsabilidad & Análisis de información, arquitectura y programación del Software  \\
\hline
Información de contacto & Toninho.monroy.ulloa@utelvt.edu.ec \\
\hline

\end{tabular}
\vspace{10pt}

\begin{tabular}{|c|c|}

\hline
Nombre & Jordie Jorell Aguilar Martinez \\
\hline
Rol & Analista, arquitecto y desarrollador de software \\
\hline
Responsabilidad & Análisis de información, arquitectura y programación del Software  \\
\hline
Información de contacto & Jordie.aquilar.martinez@utelvt.edu.ec \\
\hline

\end{tabular}
\vspace{10pt}

\begin{tabular}{|c|c|}

\hline
Nombre & Pianina Zambrano  \\
\hline
Rol & Analista, arquitecto y desarrollador de software \\
\hline
Responsabilidad & Análisis de información, arquitectura y programación del Software  \\
\end{tabular}
\subsection{\textbf{Definiciones, acrónimos y abreviaturas}}

\vspace{10pt}

\begin{tabular}{|c|c|}

\hline
\textit{Nombre} & \textit{Descripción}  \\
\hline
Usuario & Persona que usará el sistema para gestionar asignaturas \\
\hline
SGAI & Sistema de Gestión de Portafolios Integrado \\
\hline
ERS & Especificación de Requisitos Software \\
\hline
RF & Requerimiento Funcional \\
\hline
RNF & Requerimiento No Funcional\\
\hline
FEE & Factibilidad, exactitud y eficacia \\
\hline

\end{tabular}

\subsection{\textbf{Referencias}}

\begin{tabular}{|c|c|}

\hline
\textit{Título de documento} & \textit{Referencias}  \\
\hline
Standard IEEE 830 - 1998 & IEEE-830 \\
\hline

\end{tabular}


\newpage

\subsection{\textbf{Resumen}}

Este documento sobre el software de gestión de portafolios consta
principalmente de tres secciones fundamentales. En la primera
sección, se presenta una introducción que ofrece una visión general
de la especificación de requisitos del software de gestión de 
portafolios.
La segunda sección se dedica a una descripción global del sistema,
donde se detallan minuciosamente las funcionalidades del software
para la gestión de portafolios, además de los recursos necesarios
para su ejecución eficiente. Este análisis exhaustivo del sistema es
crucial, ya que sienta las bases antes de la creación del software 
de gestión de portafolios.
En la tercera sección, se establecen con precisión los requisitos, 
tanto directos como indirectos, que el sistema de gestión de 
portafolios debe cumplir para su uso adecuado. Estos requisitos son 
fundamentales para asegurar la factibilidad, precisión y eficacia
del software durante su uso continuado en el ámbito educativo.

\section{\textbf{Descripción general}}

\subsection{\textbf{Descripción del producto}}

Al completar el desarrollo del software de gestión de portafolios, y
ejecutarse, proporcionará a los usuarios una interfaz organizada y
eficiente para gestionar y acceder a los datos de portafolios, 
incluyendo planes de estudio, comunicaciones y evaluaciones. La 
interfaz reflejará de manera precisa y coherente la estructura y la 
información tal y como se ha diseñado en la codificación del 
software.

\subsection{\textbf{Funcionalidad del Software}}


\subsection{\textbf{Características de los usuarios}}

\vspace{10pt}

\begin{tabular}{|c|c|}

\hline
Tipo de usuario  & Usuario del Software \\
\hline
Formación & Universitaria \\
\hline
Actividades & Monitorización y gestión de Portafolios \\
\hline

\end{tabular}

\subsection{\textbf{Restricciones}}

\begin{enumerate}
\item Software desarrollado para sistema operativo windows 7 en 
adelante
\item  Lenguajes y tecnologías en uso: Java, PHPMyAdmin.
\item  El sistema deberá tener un diseño e implementación sencilla.
\end{enumerate}

\subsection{\textbf{Suposiciones y dependencias}}
\begin{enumerate}
\item Se asume que los requisitos aquí descritos son estables.
\item Los equipos en los que se vaya a ejecutar el sistema deben
cumplir los requisitos antes indicados para garantizar una ejecución
correcta de la misma.
\end{enumerate}

\section{\textbf{Descripción general}}

\subsection{\textbf{Requerimientos Funcionales}}


\vspace{15pt}

\begin{tabular}{|c|c|} 
\hline
\textbf{Identificación del requerimiento:} & RF01  \\
\hline
\textbf{Nombre del Requerimiento}: & Visualizar el contenido de cada 
uno de los portafolios \\
\hline
\textbf{Características:} & Permitirá únicamente la visualización de 
todas las tareas y \\ & actividades que se carguen en un portafolio
específico.  \\
\hline
\textbf{Descripción del requerimiento:} & El sistema podrá brindar 
datos de visualización\\ & al usuario autorizado para que constate 
la actividad de los portafolios \\
\hline
\textbf{Requerimiento NO funcional:} & RNF01 \\
& RNF04  \\
\hline
\textbf{Prioridad de requerimiento:} & Alta \\
\hline

\end{tabular}



\vspace{15pt}

\begin{tabular}{|c|c|} 
\hline
\textbf{Identificación del requerimiento:} & RF02  \\
\hline
\textbf{Nombre del Requerimiento}: &Visualizar Guía de estudio y 
Sílabo \\
\hline
\textbf{Características:} & Permitirá únicamente la visualización de
la guía de  estudio y el\\ & sílabo que  se carguen en una 
asignatura específica.  \\
\hline
\textbf{Descripción del requerimiento:} & El sistema podrá brindar 
datos de visualización al \\ & usuario autorizado para que 
monitorice si los documentos cargados \\ & en esa asignatura son los 
adecuados \\
\hline
\textbf{Requerimiento NO funcional:} & RNF01 \\
\hline
\textbf{Prioridad de requerimiento:} & Alta \\
\hline

\end{tabular}


\vspace{15pt}

\begin{tabular}{|c|c|} 
\hline
\textbf{Identificación del requerimiento:} & RF03  \\
\hline
\textbf{Nombre del Requerimiento}: & Agregar asignatura nueva \\
\hline
\textbf{Características:} & Le brindará al usuario un formulario que
deberá llenar  con \\ & los datos de la nueva asignatura que desea 
agregar  \\
\hline
\textbf{Descripción del requerimiento:} & Permitirá que el 
administrador de Portafolios añada  un\\ & portafolio  nueva dentro
del sistema educativo \\
\hline
\textbf{Requerimiento NO funcional:} & RNF01 \\
& RNF04  \\
\hline
\textbf{Prioridad de requerimiento:} & Alta \\
\hline

\end{tabular}


\vspace{15pt}

\begin{tabular}{|c|c|} 
\hline
\textbf{Identificación del requerimiento:} & RF04  \\
\hline
\textbf{Nombre del Requerimiento}: & Editar asignatura \\
\hline
\textbf{Características:} & Le brinda al usuario el formulario 
necesario para editar los datos de \\ &una asignatura ya agregada \\
\hline
\textbf{Descripción del requerimiento:} & Por medio de un formulario
se cargan los datos ya registrados en\\ & una asignatura específica
con la posibilidad de editar los mismos \\
\hline
\textbf{Requerimiento NO funcional:} & RNF01  \\
\hline
\textbf{Prioridad de requerimiento:} & Alta \\
\hline

\end{tabular}

\vspace{15pt}

\begin{tabular}{|c|c|} 
\hline
\textbf{Identificación del requerimiento:} & RF05  \\
\hline
\textbf{Nombre del Requerimiento}: & Eliminación de estudiantes \\
\hline
\textbf{Características:} & Le otorgará privilegios al usuario para
eliminar los estudiantes \\ & ue se encuentren matriculados en una 
asignatura \\
\hline
\textbf{Descripción del requerimiento:} & Mediante una previa 
consulta de los estudiantes\\ & que se encuentren matriculados en
una determinada \\ &  asignatura el usuario podrá eliminar \\
\hline
\textbf{Requerimiento NO funcional:} & RNF01 \\
& RNF02  \\
\hline
\textbf{Prioridad de requerimiento:} & Alta \\
\hline

\end{tabular}


\vspace{15pt}

\begin{tabular}{|c|c|} 
\hline
\textbf{Identificación del requerimiento:} & RF06  \\
\hline
\textbf{Nombre del Requerimiento}: & Agregar nuevo estudiante a 
asignatura \\
\hline
\textbf{Características:} & Permitirá que el usuario agregue un 
nuevo \\ & estudiante en una determinada asignatura \\
\hline
\textbf{Descripción del requerimiento:} & Mediante una previa
consulta de los estudiantes que\\ & se encuentren registrados en el
plantel educativo, \\ &  el usuario podrá agregarlos a cualquier 
asignatura que desee \\
\hline
\textbf{Requerimiento NO funcional:} & RNF01 \\
& RNF02  \\
\hline
\textbf{Prioridad de requerimiento:} & Alta \\
\hline

\end{tabular}


\vspace{15pt}

\begin{tabular}{|c|c|} 
\hline
\textbf{Identificación del requerimiento:} & RF07  \\
\hline
\textbf{Nombre del Requerimiento}: & Eliminación de estudiantes \\
\hline
\textbf{Características:} & Le otorgará privilegios al usuario para 
eliminar los estudiantes \\ &que se encuentren matriculados en una
asignatura \\
\hline
\textbf{Descripción del requerimiento:} & Mediante una previa 
consulta de los estudiantes que se\\ & encuentren matriculados en
una determinada \\ &  asignatura el usuario podrá eliminar \\
\hline
\textbf{Requerimiento NO funcional:} & RNF01 \\
& RNF03  \\
\hline
\textbf{Prioridad de requerimiento:} & Alta \\
\hline

\end{tabular}


\vspace{15pt}

\begin{tabular}{|c|c|} 
\hline
\textbf{Identificación del requerimiento:} & RF08  \\
\hline
\textbf{Nombre del Requerimiento}: & Configuración del perfil de
usuario \\
\hline
\textbf{Características:} & Permitirá que se modifiquen datos de los 
perfiles autorizados \\ & para esta gestión de asignaturas \\
\hline
\textbf{Descripción del requerimiento:} & Le otorgará al usuario el
privilegio de editar \\ &los datos personales de su perfil como 
nombre, foto, contraseña, \\ & o descargar una copia de seguridad \\
\hline
\textbf{Requerimiento NO funcional:} & RNF01 \\
\hline
\textbf{Prioridad de requerimiento:} & Alta \\
\hline

\end{tabular}

\subsection{\textbf{Requerimientos No Funcionales}}

\vspace{15pt}

\begin{tabular}{|c|c|} 
\hline
\textbf{Identificación del requerimiento:} & RNF01  \\
\hline
\textbf{Nombre del Requerimiento}: & Interfaz sencilla e intuitiva
para el usuario \\
\hline
\textbf{Características:} & Interfaz sencilla e intuitiva para el
usuario\\ & para que sea de fácil manejo a los usuarios del sistema.
\\
\hline
\textbf{Descripción del requerimiento:} & El sistema debe tener una
interfaz de uso intuitiva y sencilla.  \\
\hline
\textbf{Prioridad de requerimiento:} & Alta \\
\hline

\end{tabular}


\vspace{15pt}

\begin{tabular}{|c|c|} 
\hline
\textbf{Identificación del requerimiento:} & RNF02  \\
\hline
\textbf{Nombre del Requerimiento}: & Consulta de Estudiantes \\
\hline
\textbf{Características:} & Permite que el usuario consulte los
estudiantes\\
\hline
\textbf{Descripción del requerimiento:} & Mediante un pequeño
formulario el software permite lque el usuario \\ & pueda consultar
desde la  base de datos los  estudiantes matriculados \\
\hline
\textbf{Prioridad de requerimiento:} & Alta \\
\hline

\end{tabular}

\vspace{15pt}

\begin{tabular}{|c|c|} 
\hline
\textbf{Identificación del requerimiento:} & RNF03  \\
\hline
\textbf{Nombre del Requerimiento}: & Notas de los estudiantes \\
\hline
\textbf{Características:} & Permite que se genere un reporte de las
notas de los estudiantes \\ & matriculados en una asignatura 
especifica.\\
\hline
\textbf{Descripción del requerimiento:} & Mediante un pequeño
formulario el software permite  que el usuario\\ & pueda consultar
desde la base de datos  las notas\\ & de los estudiantes 
matriculados en una asignatura especifica \\
\hline
\textbf{Prioridad de requerimiento:} & Alta \\
\hline

\end{tabular}

\vspace{15pt}

\begin{tabular}{|c|c|} 
\hline
\textbf{Identificación del requerimiento:} & RNF04  \\
\hline
\textbf{Nombre del Requerimiento}: & Calendario académico \\
\hline
\textbf{Características:} & Otorga información sobre las fechas establecidas para las tareas y \\ & actividades de los portafolios.\\
\hline

\textbf{Descripción del requerimiento:} & En el calendario se reflejará con fecha exacta las actividades de cada\\ & asignatura, como tareas, y eventos educativos. \\
\hline
\textbf{Prioridad de requerimiento:} & Alta \\
\hline
\end{tabular}

\end{tabular}

\vspace{15pt}

\begin{tabular}{|c|c|} 
\hline
\textbf{Identificación del requerimiento:} & RNF05  \\
\hline
\textbf{Nombre del Requerimiento}: & Seguridad en información. \\
\hline
\textbf{Características:} & El Sistema garantizara a los usuarios \\ & una seguridad en cuanto a la información que se procede en el sistema. \\
\hline
\textbf{Descripción del requerimiento:} & Garantizar la seguridad del sistema con respecto a la información \\ & datos que se manejen tales sean documentos, archivos y contraseñas. \\
\hline
\textbf{Prioridad de requerimiento:} & Alta \\
\hline

\end{tabular}

\subsection{\textbf{Requisitos comunes de las interfaces}}



\subsubsection{Interfaces de usuario}
La interfaz con el usuario consistirá en un conjunto de ventanas con botones, listas y campos de textos. Esta deberá ser construida específicamente para el sistema propuesto y, será visualizada desde un navegador de internet.

\subsubsection{Interfaces de hardware}

Sera necesario disponer de equipos de cómputos en perfecto estado con las siguientes características.

\begin{enumerate}
\item Adaptadores de red.
\item Procesador de 1.66GHz o superior.
\item Memoria mínima de 256Mb.
\item Mouse.
\item Teclado.
\end{enumerate}


\subsubsection{Interfaces de software}
\item Sistema Operativo: Windows XP o superior.
\item Explorador: Mozilla o Chrome.


\subsubsection{Interfaces de comunicación}
Los servidores y aplicaciones se comunicaran entre sí, mediante protocolos estándares en internet, siempre que sea posible. Por ejemplo, para transferir archivos o documentos deberán utilizarse protocolos existentes (FTP u otros convenientes).

\subsection{\textbf{Requisitos funcionales}}

\subsubsection{Requisito funcional 1}

Autentificación de Usuarios: los usuarios deberán identificarse para acceder a cualquier parte del sistema.


\subsubsection{Requisito funcional 2}

Registrar Usuarios: El sistema permitirá al usuario (Administrador) registrarse. El usuario debe suministrar datos como: Nombre, Apellido, Email, Usuario y Password.

\subsubsection{Requisito funcional 3}

Modificar: Permite al administrador modificar la contraseña de entrada o algún archivo mal seleccionado.
\end{tabular}
\begin{document}
\subsection{\textbf{Requisitos no funcionales}}
\begin{document}

\subsubsection{Requisitos de rendimiento}

identificar y medir el riesgo asociado con cada inversión en el portafolio y tomar medidas para mitigarlo en la medida de lo posible. Los gestores de carteras deben establecer límites de riesgo y trabajar en estrategias de gestión de riesgos adecuadas para garantizar que los riesgos no superen los límites establecidos y que estén alineados con los objetivos de inversión y la tolerancia al riesgo de los inversores.

\subsubsection{Seguridad}
\begin{itemize}
    \item Implementar sistemas robustos de autenticación de usuarios y control de acceso basados en roles para garantizar que solo las personas autorizadas tengan acceso a la información confidencial y a las funciones críticas del portafolio.
    
    \item Utilizar protocolos de encriptación sólidos para proteger la integridad de los datos confidenciales y sensibles, tanto en reposo como en tránsito. La encriptación ayuda a evitar el acceso no autorizado y la manipulación de datos por parte de terceros no deseados.
   
    \item Establecer un sistema de registro y seguimiento detallado de todas las actividades relacionadas con la gestión de portafolios. La capacidad de realizar un seguimiento de las transacciones, las modificaciones y los accesos ayuda a detectar y prevenir cualquier actividad sospechosa o no autorizada de manera oportuna.
\end{itemize}

\subsubsection{Fiabilidad}

Asegurarse de que el sistema de gestión de portafolios sea capaz de manejar grandes volúmenes de datos y transacciones de manera eficiente y sin interrupciones. La capacidad de escalar según las necesidades del negocio y de proporcionar un rendimiento constante y de alta calidad es fundamental para garantizar la fiabilidad del sistema

\subsubsection{Disponibilidad}

Contar con una infraestructura de TI robusta que pueda soportar cargas de trabajo intensivas y mantener la disponibilidad del sistema sin interrupciones. Esto implica el uso de servidores confiables, almacenamiento de datos redundante y redes de alta velocidad para garantizar un acceso rápido y seguro a los datos del portafolio.

\subsubsection{Mantenibilidad}

\begin{itemize}
    \item Implementar un programa sólido de mantenimiento preventivo y correctivo para identificar y solucionar problemas potenciales del sistema de gestión de portafolios de manera oportuna. Además, ofrecer un soporte técnico eficaz para abordar cualquier problema que pueda surgir y garantizar que el sistema funcione de manera confiable y sin problemas para los usuarios finales.
   
    \item Establecer sistemas de monitorización en tiempo real para supervisar continuamente el rendimiento y la disponibilidad del sistema de gestión de portafolios. Esto permite identificar y abordar cualquier problema potencial antes de que afecte significativamente la disponibilidad del sistema y el acceso de los usuarios.
\end{itemize}

\subsubsection{Portabilidad}

Asegurarse de que el sistema de gestión de portafolios sea compatible con diferentes plataformas y entornos, como sistemas operativos y dispositivos móviles. La capacidad de funcionar de manera óptima en una variedad de entornos garantiza que los usuarios puedan acceder y gestionar sus carteras de inversiones desde diferentes dispositivos y ubicaciones. 

\end{tabular}
\end{document}
\end{document}
